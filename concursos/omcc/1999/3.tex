 Las cifras de una calculadora (a excepción del $0$) están dispuestas en la forma indicada en el cuadro adjunto, donde aparece también la tecla '$+$'.
    \begin{center}
    \begin{tabular}{|c|c|c|c|}
        \hline
        7&8&9&\\
        \hline
        4&5&6&+\\
        \hline
        1&2&3&\\
        \hline
    \end{tabular}
    \end{center}
    Dos jugadores $A$ y $B$ juegan de la manera siguiente: $A$ enciende la calculadora y pulsa una cifra, y a continuación pulsa la tecla $+$. Pasa la calculadora a $B$, que pulsa una cifra en la misma fila o columna que la pulsada por $A$ que no sea la misma que la última pulsada por $A$; a continuación pulsa $+$ y le devuelve la calculadora a $A$, que repite la operación y así sucesivamente. Pierde el juego el primer jugador que alcanza o supera la suma $31$. ¿Cuál de los dos jugadores tiene una estrategia ganadora y cuál es esta?
