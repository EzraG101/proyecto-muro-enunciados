\documentclass[11pt]{scrartcl}
\usepackage{muro}
\usepackage[inline]{asymptote}


%%fancyhdr
\pagestyle{fancy}
\lfoot{\sffamily\bfseries\hyperlink{tabla}{Índice}}
\rfoot{\thepage}
\cfoot{}
\lhead{\color{jampmDate}{\sffamily\today}}
\rhead{\rmfamily{Proyecto MURO}}
%name of the handout, create variable with that string package on the macros page in wikibook
\chead{{\sffamily\large Compilación OMCC}}

%HyperSETUP
\hypersetup{
colorlinks= true,
urlcolor= cyan,
linkcolor= jampmLinks,
citecolor=red,
pdftitle={Compilación OMCC},
bookmarks = true,
pdfpagemode = FullScreen,
}


\newenvironment{problema}{
\stepcounter{Problemas}
\noindent{\bfseries\sffamily\large Problema \theProblemas.}
}{


\vspace{15pt}


}
\begin{document}
\newcounter{Problemas}[subsection]

\title{Compilación OMCC}
\author{cosa 1, 2 y 3}
\date{\today}
\maketitle
\epigraph{Baby la vida es un ciclo hamiltoniano}{Juan}
\noindent{\bfseries\Large\sffamily Introducción}

\noindent Somos el proyecto MURO, y nuestro objetivo es conformar un Movimiento Unificado de Recursos Olímpicos (el nombrecito es un meme ngl). Esta es una compilación con todos los problemas de la Olimpiada de Matemáticas de Centroamérica y el Caribe, con enunciados, pistas y soluciones. Esperamos la disfrutes! y no dudes en compartirnos cualquier comentario o crítica constructiva. Nos puedes contactar por aquí: [insert contacto]. 
\hypertarget{tabla}{\tableofcontents}
\vfill
\eject

\section{Problemas}

\foreach \j in {1999,2000,...,2010} {%
    \subsection{\j}
    \foreach \i in {1,2,...,6} {%
        \begin{problema}
            \enunciado{\j}{centro}{\i}
            %\sacapistas{\j}{centro}{\i}
        \end{problema}
    }
    \eject
}
\eject

\section{Pistas}

\makeatletter
\def\declarenumlist#1#2#3{%
\expandafter\edef\csname pgfmath@randomlist@#1\endcsname{#3}%
\count@\@ne
\loop
\expandafter\edef
\csname pgfmath@randomlist@#1@\the\count@\endcsname
  {\the\count@}
\ifnum\count@<#3\relax
\advance\count@\@ne
\repeat}

\declarenumlist{hintlist}{1}{\value{hintcounter}}

\def\prunelist#1{%
\expandafter\edef\csname pgfmath@randomlist@#1\endcsname
    {\the\numexpr\csname pgfmath@randomlist@#1\endcsname-1\relax}
\count@\pgfmath@randomtemp
\loop
\expandafter\let
\csname pgfmath@randomlist@#1@\the\count@\expandafter\endcsname
\csname pgfmath@randomlist@#1@\the\numexpr\count@+1\relax\endcsname
\ifnum\count@<\csname pgfmath@randomlist@#1\endcsname\relax
\advance\count@\@ne
\repeat}
\makeatother
    %% Print the hints
\begin{enumerate}
\small
\itemsep2pt
\setcounter{hindex}{0}%
\whiledo{\value{hindex} < \value{hintcounter}}{%
 \addtocounter{hindex}{1}%
 \pgfmathrandomitem\z{hintlist}
 \gethint{\z}
 \prunelist{hintlist}
}
\end{enumerate}

\eject

\section{soluciones}

\end{document}
