Se supone que $5$ personas conocen, cada una, informaciones parciales diferentes sobre cierto asunto. Cada vez que la persona $A$ telefonea a la persona $B$, $A$ le da a $B$ toda la información que conoce en ese momento sobre el asunto, mientras que $B$ no le dice nada de él. ¿Cuál es el mínimo número de llamadas necesarias para que todos lo sepan todo sobre el asunto? ¿Cuántas llamadas son necesarias si son $n$ personas?
