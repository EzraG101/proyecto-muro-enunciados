Tenemos un polígono regular $P$ con 2019 vértices, y en cada vértice hay una moneda. Dos jugadores \textit{Azul} y \textit{Rojo} se turnan alternativamente, empezando por Azul, de la siguiente manera: primero, Azul elige un triángulo con vértices en $P$ y colorea su interior de azul, luego Rojo elige un triángulo con vértices en $P$ y colorea su interior de rojo, de manera que los triángulos formados en cada jugada no se cruzan internamente con los triángulos coloreados anteriores. Continúan jugando hasta que no sea posible elegir otro triángulo para colorear. Entonces, un jugador gana la moneda de un vértice si coloreó la mayor cantidad de triángulos incidentes en ese vértice (si las cantidades de triángulos coloreados con azul o rojo incidentes en el vértice son iguales, entonces nadie gana esa moneda y la moneda se borra). El jugador con la mayor cantidad de monedas gana la partida.  Encuentra una estrategia ganadora para uno de los jugadores. \\\\
\texit{Nota.} Dos triángulos pueden compartir vértices o lados.
